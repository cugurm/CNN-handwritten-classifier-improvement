\documentclass{article}

% if you need to pass options to natbib, use, e.g.:
%     \PassOptionsToPackage{numbers, compress}{natbib}
% before loading neurips_2019

% ready for submission
% \usepackage{neurips_2019}

% to compile a preprint version, e.g., for submission to arXiv, add add the
% [preprint] option:
%     \usepackage[preprint]{neurips_2019}

% to compile a camera-ready version, add the [final] option, e.g.:
\usepackage[final]{neurips_2019}

% to avoid loading the natbib package, add option nonatbib:
%     \usepackage[nonatbib]{neurips_2019}

\usepackage[utf8]{inputenc} % allow utf-8 input
\usepackage[T1]{fontenc}    % use 8-bit T1 fonts
\usepackage{hyperref}       % hyperlinks
\usepackage{url}            % simple URL typesetting
\usepackage{booktabs}       % professional-quality tables
\usepackage{amsfonts}       % blackboard math symbols
\usepackage{nicefrac}       % compact symbols for 1/2, etc.
\usepackage{microtype}      % microtypography

\title{Learn the Way We Write: Automatic Adjustment of %offline 
Handwriting Neural Classifier to Individual Users}

% The \author macro works with any number of authors. There are two commands
% used to separate the names and addresses of multiple authors: \And and \AND.
%
% Using \And between authors leaves it to LaTeX to determine where to break the
% lines. Using \AND forces a line break at that point. So, if LaTeX puts 3 of 4
% authors names on the first line, and the last on the second line, try using
% \AND instead of \And before the third author name.

\author{%
  Milan M.~Čugurović\thanks{poincare.matf.bg.ac.rs/$\sim$milan\_cugurovic} \\
  Department of Computer Science\\
  Faculty of Mathematics\\
  University of Belgrade\\
  Belgrade, 11000\\
  Serbia\\
  \texttt{milan\_cugurovic@math.rs} \\
  % examples of more authors
  \And
  Mladen Nikolić\thanks{poincare.matf.bg.ac.rs/$\sim$nikolic} \\
  Department of Computer Science\\
  Faculty of Mathematics\\
  University of Belgrade\\
  Belgrade, 11000\\
  Serbia\\
  \texttt{nikolic@math.rs}
  \And
  Novak Novaković \\
  Microsoft Development Center Serbia\\
  Belgrade, 11000\\
  Serbia\\
  % \AND
  % Coauthor \\
  % Affiliation \\
  % Address \\
  % \texttt{email} \\
  % \And
  % Coauthor \\
  % Affiliation \\
  % Address \\
  % \texttt{email} \\
  % \And
  % Coauthor \\
  % Affiliation \\
  % Address \\
  % \texttt{email} \\
}

\begin{document}

\maketitle

\begin{abstract}
  In this paper, we consider improving convolution neural network (CNN) classifier of offline handwritten text.  
  We focus on the style of handwriting each of the individual users, causing of its great variability and huge impact on the ability to successfully recognize written characters. 
  We present fast, scalable and a no-retrain method for improving CNNs classifier. 
  Using only basic machine learning techniques, such as K nearest neighbor classifier and K means clustering, we achieve up to +2.7\% improvement in neural classifier precision, and get state of the art results on the dataset we use. 
  We evaluate our method on two dataset: NIST Special Database 19 and ETH Zurich Deepwriting dataset. 
\end{abstract}

\section{Introduction}
The problem of automatic recognition of offline handwritten characters is a very practical problem, which is a part of the area of pattern recognition. 
Inspired by practical use, it develops both within academia as well as within the industrial sector. 
The industrial sector directly commercializes solutions of this problem by including them into devices like tablets, smartphones and the like. 
That is why it is very important to have precise classifiers that rarely make mistakes. 

\section{Related work}
pass 

\subsection{Previous work in offline handwritten character recognition}
pass

\subsection{Previous work in improving offline handwriting classifiers}
pass

\section{Method}
pass

\subsection{Method overview}
pass 

\subsection{Clustering individual character writing styles}
pass 

\subsection{Creating a writing history}
Možda zaista usvojiti naziv 'dynamic writing history' za istoriju pisanja?

\subsection{Using writing history}
pass

%\section{Implementation}
%pass

\section{Evaluation}
pass

\subsection{Used datasets}
pass

\subsubsection{NIST Special Database 19}
pass

\subsubsection{ETH Zurich Deepwriting Database}
pass

\subsection{Images preprocessing}
pass

\subsection{Datasets split}
pass

\subsubsection{NIST Special Database 19 split}
pass \citet{nist}

\subsubsection{ETH Zurich Deepwriting Database split}
pass \cite{deepwriting}

\subsection{Base CNN classificator}
pass

\subsubsection{Architecture}
pass 

\subsubsection{Training and results}
pass

\subsection{Evaluation results}
pass

\subsection{Comparison with relevant papers}
Ovo možda u related works?

\section{Conclusion}
pass 

\small

\begin{thebibliography}{99} 
  \bibitem[Patrick et al. (2016)]{nist} Patrick, Grother \ \&  Kayee, Hanaoka\ (2016) NIST Special Database 19 Handprinted Forms and Characters, 2nd Edition
  \bibitem[Aksan et al. (2018)]{deepwriting} Aksan, Emre \ \& Pece, Fabrizio \ \& Hilliges, Otmar\ (2018) DeepWriting: Making Digital Ink Editable via Deep Generative Modeling
  In {\itshape SIGCHI Conference on Human Factors in Computing Systems}: CHI '18, ACM, New York, NY, USA
\end{thebibliography}


\end{document}